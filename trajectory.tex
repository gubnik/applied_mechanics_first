\section{Траектория}
Для нахождения траектории необходимо выразить $y(x)$ через
выражение $t(x)$ из $x(t)$ и подстановку в $y(t)$.

Выразим $t(x)$:\\
\begin{align}
    x(t) &= -4t^{2} + 1 \Rightarrow -4t^{2} = x - 1 \\
    t(x) &= \pm\sqrt{\frac{1-x}{4}} = \pm\frac{\sqrt{1 - x}}{2}
\end{align}

Так как отрицательное время не имеет физического смысла, берем
положительный корень $t(x) = \frac{\sqrt{1 - x}}{2}$.
Отсюда, выразим $y(t(x)) = y(x)$:\\
\begin{align}
    y(t(x)) &= 8 - 3t(x) \Rightarrow
    y(x) = 8 - \frac{3\sqrt{1 - x}}{2}
\end{align}

\begin{figure}[ht]
    \centering
    \begin{tikzpicture}
        \begin{axis}[
            title = {Траектория $y(x)$},
            xlabel = {$x$},
            ylabel = {$y$},
            grid = major,
            xmin = -10, xmax = 1,
            ymin = 0, ymax = 15,
            domain=-10:1,
            samples=100,
            ]
            \addplot[blue, thick]{8 - (3/2)*sqrt(1 - x)};
            \addplot[dotted, thick]{8};
        \end{axis}
    \end{tikzpicture}
\end{figure}


