\section{Декартовые координаты}
В условии предоставлены уравнения движения в проекциях на оси $OX$ и $OY$
в двумерном пространстве $OXY$. Соответственно, векторные величины для
скорости $\vec{v}$ и ускорения $\vec{a}$ следует находить как вектор
формата $\vec{i}=\vec{(x_{t}, y_{t})}$, где $x_{t}$ и $y_{t}$ есть значения,
полученные для осей $OX$ и $OY$.

\subsection{Скорость}
Скорость, как известно из курса физики, есть первая произодная по времени.
Так как над даны уравнения движения для осей $x$ и $y$, то скорость будет
определяться как $$ \vec{v} = \vec{(\frac{dx}{dt}, \frac{dy}{dt})} $$
Найдем производные:
\begin{align}
    &\frac{dx}{dt} = \frac{1}{dt}(-4t^{2} + 1) \qquad
    &\frac{dx}{dt} = -8t \\
    &\frac{dy}{dt} = \frac{1}{dt}(8 - 3t) \qquad
    &\frac{dy}{dt} = -3 \\
    &\frac{dx}{dt} \bigg|_{t=0.5} = -8 \cdot 0.5 = -4 \\
    &\frac{dy}{dt} \bigg|_{t=0.5} = -3
\end{align}
Итак, вектор скорости точки равен $$ \vec{v} = ( -4, -3 ) $$,
модуль скорости равен $$ \left|{v}\right| = \sqrt{-4^{2} + -3^{2}} = 5 $$

\subsection{Ускорение}
Ускорение есть вторая производная по времени. Соответственно,
$$ \vec{a} = \vec{(\frac{d^{2}x}{dt^{2}}, \frac{d^{2}y}{dt^{2}})} $$
Найдем производные:
\begin{align}
    &\frac{d^{2}x}{dt^{2}} = \frac{1}{dt^{2}}(-4t^{2} + 1) \qquad
    &\frac{d^{2}x}{dt^{2}} = -8 \\
    &\frac{d^{2}y}{dt^{2}} = \frac{1}{dt^{2}}(8 - 3t) \qquad
    &\frac{d^{2}y}{dt^{2}} = 0 \\
    &\frac{d^{2}x}{dt^{2}} \bigg|_{t=0.5} = -8 \\
    &\frac{d^{2}y}{dt^{2}} \bigg|_{t=0.5} = 0
\end{align}
Итак, вектор ускорения точки равен $$ \vec{a} = ( -8, 0 ) $$,
модуль ускорения равен $$ \left|{a}\right| = \sqrt{-8^{2} + 0^{2}} = 8 $$

\subsection{Траектория}
Для нахождения траектории необходимо выразить $y(x)$ через
выражение $t(x)$ из $x(t)$ и подстановку в $y(t)$.

Выразим $t(x)$:\\
\begin{align}
    x(t) &= -4t^{2} + 1 \Rightarrow -4t^{2} = x - 1 \\
    t(x) &= \pm\sqrt{\frac{1-x}{4}} = \pm\frac{\sqrt{1 - x}}{2}
\end{align}

Так как отрицательное время не имеет физического смысла, берем
положительный корень $t(x) = \frac{\sqrt{1 - x}}{2}$.
Отсюда, выразим $y(t(x)) = y(x)$:\\
\begin{align}
    y(t(x)) &= 8 - 3t(x) \Rightarrow
    y(x) = 8 - \frac{\sqrt{1 - x}}{2}
\end{align}

\begin{figure}[ht]
    \centering
    \begin{tikzpicture}
        \begin{axis}[
            title = {Траектория $y(x)$},
            xlabel = {$x$},
            ylabel = {$y$},
            grid = major,
            xmin = -10, xmax = 1,
            ymin = 0, ymax = 15,
            domain=-10:1,
            samples=100,
            ]
            \addplot[blue, thick]{8 - (3/2)*sqrt(1 - x)};
            \addplot[dotted, thick]{8};
        \end{axis}
    \end{tikzpicture}
\end{figure}

\subsection{Кривизна траектории}
Кривизна траектории есть мера того, насколько траектория отклоняется от
прямолинейного движения:
\begin{align}
    k &= \frac{\left|\vec{r}'(t)\times\vec{r}''(t)\right|}{\left|\vec{r}'(t)\right|^{3}},
    \quad\vec{r}(t) - \text{уравнение движения точки}\\
\end{align}
Для двумерного пространства $OXY$ $\vec{r}(t)$ распадается на компоненты $x(t)$ и $y(t)$,
соответственно выражение принимает вид:
$$k = \frac{\left|x'y'' - y'x''\right|}{(x'^{2}+y'^{2})^{\frac{3}{2}}}$$
Из пунктов $2.1$ и $2.2$ имеем значения производных $x$ и $y$.
Подставив, получаем $k = \frac{24}{125} = 0.192$
