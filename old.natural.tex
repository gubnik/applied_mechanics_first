\section{Естественные координаты}

\subsection{Заданные функции и их значения при  $t=0.5$:}
\subsubsection{Координаты точки на траектории при $t = 0.5$}

\[
    x(0, 5) = -4  \cdot (0, 5)^{ 2} + 1 = -4 \cdot 0,25 + 1 = -1 + 1 = 0
\]

\[
    y(0.5)=8−3 \cdot (0.5)=8 − 1,5= 6,5
\]

Таким образом, в момент времени 
$t=0.5$ материальная точка находится в точке 
$(0;6,5)$

\subsubsection{Производные координат (компоненты скорости):}

\[
    {\dot x}(t) = \frac{ d (-4t^{ 2}+1)}{ d t} = -8t
\]

\[
    {\dot y(t)} =   \frac{ d (8-3t)}{ d t} = -3
\]

При $t=0.5$:
\[
 {\dot x}(0.5) = -8 \cdot 0.5 = -4
 \]


 \[
    {\dot y}(0.5) = -3
 \]



\subsubsection{Скорость (модуль вектора $\vec{v}$):}

\[
 S(t) = \sqrt{{\dot x}(t)^{ 2} + {\dot y}(t)^{ 2}}
\]

При $t=0.5$:

\[
 (0.5) = \sqrt{(-4)^{ 2} + (-3)^{ 2}} = \sqrt{16 + 9} = \sqrt{25} = 5
\]

\subsection{Ускорение и разложение его на тангенциальную и нормальную составляющие}

\subsubsection{Ускорение $\vec{a}(t)$:}

\[
 {\ddot x}(t) = \frac{ d (-8t)}{ dt} = -8
 \]
 
 \[
    {\ddot y}(t) = \frac{ d (-3)}{ d t} =0
 \]

Значит,
\[
 \vec{a}(t) = \left( {\ddot x(t), {\ddot y(t)}}\right) = (-8, 0)
\]

При $t = 0.5$:
\[
\vec{a} (0.5) = (-8 ,0)
\]

\[
    \left| \vec{a}(0.5)\right| = \sqrt{(-8)^{ 2} + 0^2} = 8
\]


\[
S = \int_{ 0}^{ t} \sqrt{ {\dot x}(t) + {\dot y}(t) }
\]

\[
    S= \int_{ 0}^{t} \sqrt{64t^{ 2} +9}dt
\]
\[
    S= \int_{ 0}^{0.5} \sqrt{64t^{ 2} +9}dt = \sqrt{25} - \sqrt{9} = 2
\]

\[
 v^{ \tau} = \sqrt{Vx(t)^{ 2} + Vy(t)^{ 2}} = \sqrt{(-4t^{2} + 1)^{ 2} + (8 - 3t)^{ 2}}
\]
    % \(
% \)

% \(
%  {\dot S} = \sqrt{ {\dot x^{ 2}} + {\dot y^{ 2}}} = 
% \)

% \(
%  \vec{a} = {\ddot s} \vec{\tau} + \frac{{\dot S}^{ 2}  }{ \rho} \vec{n}
% \) 

% \(
%  {\ddot S} = \frac{ {\dot x} {\ddot x} + {\dot y} {\ddot y} }{ \nu} =
% \)

% \(
%  \left| \vec{a}\right| = \sqrt{\ddot{S}^{ 2} + \frac{{\dot S}^{ 4}}{\rho^{ 2}}} =
% \) 

% \(
%  \rho = \sqrt{\frac{  {\dot S}^{ 4}  }{ \left| \vec{a}\right|^{ 2} - \vec{S}^{ 2}}} =
% \) 

% \(
%  \vec{a} = 
% \)

% \(
%  a = 
% \)
